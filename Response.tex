\documentclass[11pt]{article}
\usepackage{geometry}
\geometry{letterpaper, margin=1in}
\usepackage{color}
\usepackage{enumitem}
\usepackage{amsmath}
\usepackage{amssymb}
\usepackage{hyperref}

\definecolor{darkblue}{rgb}{0,0,0.8}

\newcommand{\reviewercomment}[1]{\textbf{Reviewer Comment:} \textit{#1}}
\newcommand{\response}[1]{\textbf{Response:} #1}
\newcommand{\changes}[1]{\textbf{Changes Made:} #1}

\title{\Large Response to Reviewer Comments\\
\vspace{0.5em}
\large Manuscript: Robust Optimization via Continuous-Time Dynamics}
\author{}
\date{\today}

\begin{document}
\maketitle

\noindent Dear Editor and Reviewers,

We sincerely thank the reviewers for their valuable comments and suggestions. We have carefully addressed all concerns raised and made substantial revisions to improve the manuscript. All changes in the revised manuscript are marked in {\color{blue}blue} for easy identification. Below, we provide detailed point-by-point responses to each reviewer's comments.

\section*{Response to Reviewer 4}

\subsection*{Comment 1: Unclear Improvements Over Existing Results}

\reviewercomment{The improvements compared to the existing results are also unclear, which makes it difficult to evaluate the contribution of this article.}

\response{We agree that the contributions needed to be more explicitly stated. We have added two new subsections in the Introduction to clearly articulate our contributions and improvements over existing methods.}

\changes{
\begin{itemize}
\item Added Section I.A ``Main Contributions'' listing 6 key contributions including:
  \begin{enumerate}
  \item Model-free approach without requiring a priori problem knowledge
  \item Unified framework for general convex-concave robust optimization
  \item Novel dynamical system design different from classical primal-dual methods
  \item Custom Lyapunov function for global convergence analysis
  \item Real-time capability for dynamic environments
  \item Broader applicability to cases where robust counterparts cannot be derived
  \end{enumerate}
\item Added Section I.B ``Comparison with Existing Methods'' with comprehensive Table I comparing our approach to existing methods (Robust Counterpart, Scenario Sampling, Oracle-based, Cutting Plane)
\item Updated literature review with modern state-of-the-art references
\end{itemize}
}

\subsection*{Comment 2: Language and Grammar Issues}

\reviewercomment{There are some language and grammar issues in this paper and the authors need to revise their paper properly.}

\response{We have conducted a comprehensive review of the entire manuscript to fix all language and grammar issues.}

\changes{
\begin{itemize}
\item Performed thorough grammar and language corrections throughout
\item Improved clarity of technical descriptions
\item Shortened lengthy sentences for better readability
\item Replaced ``the paper'' with ``this paper'' consistently
\item Enhanced overall writing clarity and conciseness
\end{itemize}
}

\subsection*{Comment 3: Article Usage in Captions}

\reviewercomment{The definite article in the captions of all figures is suggested to be omitted. In general, the definite article in the title should be omitted.}

\response{We have removed all definite articles from figure captions and section titles as suggested.}

\changes{
\begin{itemize}
\item Figure 1-4: Removed ``The'' from all captions (e.g., ``The trajectories'' → ``Trajectories'')
\item Section titles: Removed articles (e.g., ``Saddle property of the optimal RO solution'' → ``Saddle property of optimal RO solution'')
\end{itemize}
}

\subsection*{Comment 4: Formula Italics}

\reviewercomment{The italics of many formulas in this manuscript are not standard, and there are some inconsistent phenomena.}

\response{We have reviewed all mathematical notation to ensure consistent formatting throughout.}

\changes{
\begin{itemize}
\item Ensured consistent use of \texttt{\textbackslash operatorname} for functions like min, max, argmin
\item Standardized variable formatting in all equations
\item Verified consistent mathematical notation throughout
\end{itemize}
}

\subsection*{Comment 5: Quotation Marks}

\reviewercomment{There are issues with the quotation marks in the manuscript.}

\response{We have corrected all quotation mark issues throughout the manuscript.}

\changes{Fixed all reversed or incorrect quotation marks, particularly in page 5 and throughout the document.}

\section*{Response to Reviewer 5}

\subsection*{Comment 1: Introduction Organization}

\reviewercomment{The paper discusses contributions and motivation in multiple sections, but these points are not clearly articulated or cohesively presented. To improve readability and impact, I recommend reorganizing the introduction to explicitly highlight the key contributions.}

\response{We have completely reorganized the Introduction to clearly present our contributions and motivation.}

\changes{
\begin{itemize}
\item Restructured Introduction with explicit subsections for contributions and comparisons
\item Added clear enumeration of 6 main contributions
\item Shortened introduction while maintaining key information
\item Improved logical flow and readability
\end{itemize}
}

\subsection*{Comment 2: Algorithm 23 Explanation}

\reviewercomment{Algorithm (23) is presented in isolation without sufficient explanation or context. The authors should provide a detailed discussion immediately after introducing the algorithm, including the intuition behind its design, a clear comparison with existing methods to highlight key differences, and specific improvements or advantages over traditional approaches.}

\response{We have added comprehensive explanation and context for Algorithm 23.}

\changes{
\begin{itemize}
\item Added detailed step-by-step discussion immediately after Algorithm 23
\item Included intuition behind the design choices
\item Added comparison with standard primal-dual methods
\item Explained specific advantages over traditional approaches
\item Introduced Z parameter explanation right after equation (23)
\end{itemize}
}

\subsection*{Comment 3: Convergence Performance Analysis}

\reviewercomment{The paper lacks a detailed theoretical analysis of the proposed algorithm's convergence performance compared to existing methods.}

\response{We have added comprehensive convergence analysis sections.}

\changes{
\begin{itemize}
\item Added Section V-D.1 ``Convergence Rate Analysis''
\item Added Section V-D.2 ``Stability Analysis''
\item Enhanced theoretical comparison with existing methods in Table I
\item Provided detailed convergence properties discussion
\end{itemize}
}

\subsection*{Comment 4: Equation Reference Order}

\reviewercomment{The structure of the manuscript needs improvement for better readability, as equations (5) and (6) are referenced in Assumption 2 before they are formally introduced in the text.}

\response{We have fixed the equation referencing order issue.}

\changes{Removed forward references to equations before they are defined, replacing with more general descriptions that maintain clarity without referencing undefined equations.}

\subsection*{Comment 5: Assumption 2 Restrictions}

\reviewercomment{The conditions in Assumption 2 seem restrictive. Can they be relaxed, as in Ref. [34]? If not, please provide a detailed explanation.}

\response{We have added detailed justification for Assumption 2 and discussed possible relaxations.}

\changes{Added comprehensive Remark on ``Justification of Assumption 2'' explaining:
\begin{itemize}
\item Why Slater conditions ensure strong duality
\item How the assumption can be satisfied in practice
\item Alternative constraint qualifications that could be used
\item Practical enforcement strategies
\end{itemize}
}

\subsection*{Comment 6: Assumption 1 Justification}

\reviewercomment{The need for Assumption 1 should be justified. Can the framework be extended to more general cases?}

\response{We have added justification and discussed extensions.}

\changes{Added Remark on ``Justification of Assumption 1'' explaining:
\begin{itemize}
\item Fundamental role of convex-concave structure
\item Possible relaxations (strict convexity to convexity)
\item Extensions to non-smooth cases
\item Common uncertainty sets that satisfy requirements
\end{itemize}
}

\subsection*{Comment 7: Notation Clarification}

\reviewercomment{The meaning of $h_{ij}$ and $K_i$ in equation (3) should be clearly understood, and the purpose of introducing them here should be explicitly justified.}

\response{We have added clear explanations of the notation.}

\changes{Added clarification: ``Here, $h_{ij}(u_i)$ represents the $j$-th constraint function defining the $i$-th uncertainty set $\mathcal{U}_i$, and $K_i$ denotes the total number of constraints that define the uncertainty set $\mathcal{U}_i$.''}

\subsection*{Comment 8: Lagrangian Analysis Necessity}

\reviewercomment{The Lagrangian function (15) for the RO problem (4) appears to be straightforward, raising questions about the necessity of the lengthy and intricate analysis preceding it.}

\response{We have added a comprehensive remark explaining why the detailed Lagrangian analysis is necessary.}

\changes{Added ``Remark on Necessity of Lagrangian Analysis'' explaining:
\begin{itemize}
\item The non-trivial nested min-max structure challenge
\item How unified Lagrangian enables convergence analysis
\item Why it differs from standard approaches
\item The theoretical importance for global convergence proof
\end{itemize}
}

\subsection*{Comment 9: Abbreviation Issues}

\reviewercomment{The abbreviation ``RC'' in the line before equation (9) has been explained earlier and could be deleted here. The meaning of the abbreviation ``RHS'' after inequality (51) needs to be explained.}

\response{We have fixed all abbreviation issues.}

\changes{
\begin{itemize}
\item Removed duplicate ``RC'' explanation before equation (9)
\item Added explanation for ``RHS'' (right-hand side) after inequality (51)
\end{itemize}
}

\subsection*{Comment 10: Appendix B Content}

\reviewercomment{In Section V-D, since the Appendix B only contains conclusions without detailed proofs, it is recommended to incorporate these conclusions directly into the main text for better readability and logical flow.}

\response{We have moved the relevant content from Appendix B to the main text.}

\changes{Incorporated Appendix B conclusions into Section V-D for better flow and readability.}

\subsection*{Comment 11: Missing Proof of Lemma 4}

\reviewercomment{We note that the proof of Lemma 4 is not provided in Ref. [41], and this issue needs to be addressed.}

\response{We have verified that the reference to Lemma 4 (omega invariance) is properly cited from Cherukuri et al. (2016) where it appears as Lemma 4.4 with complete proof.}

\changes{Confirmed proper citation to Cherukuri et al. (2016) Lemma 4.4 which contains the complete proof.}

\subsection*{Comment 12: Superscript Notation}

\reviewercomment{What is the meaning of the superscript $\epsilon^+$ in the second line of equation (43)? Please explain.}

\response{We have added clarification for the notation.}

\changes{Added explanation that $\epsilon^+$ denotes the positive projection operator as defined in equation (1).}

\subsection*{Comment 13: Simulation Results}

\reviewercomment{The results obtained from Ref. [22] in the simulation are much smaller, so why are the results of the proposed algorithm in this paper considered to be more effective?}

\response{We have enhanced the explanation and updated comparisons.}

\changes{
\begin{itemize}
\item Added explanation of why our approach is more effective despite different numerical values
\item Updated comparisons with modern state-of-the-art algorithms
\item Included additional performance metrics beyond simple numerical comparison
\end{itemize}
}

\subsection*{Comment 14: Proposition 6 Justification}

\reviewercomment{The justification for Proposition 6 in Appendix B requires further clarification to enhance understanding.}

\response{We have added comprehensive justification for Proposition 6.}

\changes{Added ``Remark on Projected Dynamics Justification'' explaining the necessity and importance of the projected dynamics analysis for existence, uniqueness, and continuity of solutions.}

\subsection*{Comment 15: Limit Expression}

\reviewercomment{The expression $\lim_{k \to \infty}=y$ in Proposition 6 seems incorrect. It should likely be $\lim_{k \to \infty} y_k=y$.}

\response{We have corrected the notation error.}

\changes{Fixed the expression to correctly show $\lim_{k \to \infty} y_k=y$.}

\subsection*{Comment 16: Language Issues}

\reviewercomment{There are several language issues, such as the reversed quotation marks in the last paragraph of the first column on page 5. Please carefully proofread the manuscript.}

\response{We have carefully proofread and corrected all language issues.}

\changes{Fixed all quotation marks and conducted comprehensive proofreading throughout.}

\subsection*{Comment 17: Introduction Length}

\reviewercomment{The introduction is lengthy and lacks a coherent structure. Please revise it to emphasize the advantages of the proposed method, particularly its ability to operate without prior problem modeling.}

\response{We have revised the Introduction for better structure and clarity.}

\changes{
\begin{itemize}
\item Shortened introduction while maintaining key information
\item Added clear structure with subsections
\item Emphasized model-free capabilities prominently
\item Improved coherent flow of ideas
\end{itemize}
}

\subsection*{Comment 18: Modern Algorithm Comparison}

\reviewercomment{The techniques from Ref. [22] used in the simulation appear outdated and may not reflect the current advancements in the field. To strengthen the comparative analysis, it is recommended to include state-of-the-art algorithms in the evaluation.}

\response{We have updated the comparisons with modern algorithms.}

\changes{
\begin{itemize}
\item Updated literature review with recent references
\item Added comparison with state-of-the-art methods in Table I
\item Enhanced simulation section with modern algorithm comparisons
\end{itemize}
}

\section*{Response to Reviewer 6}

\subsection*{Major Comment 1: Problem Formulation Motivation}

\reviewercomment{My concern lies in the motivation behind the problem formulation (4). Does it offer any advantages compared to formulation (3)? The authors should emphasize the main reason for introducing (4), beyond merely presenting it as a more general version of (3).}

\response{We have added comprehensive explanation of the motivation for our formulation.}

\changes{Added ``Remark on Problem Formulation Clarifications'' explaining:
\begin{itemize}
\item How the $c_i$ terms provide regularization for inactive constraints
\item Computational benefits and numerical stability advantages
\item Natural recovery of classical formulation when $c_i \to 0$
\item Flexibility in handling both active and inactive constraints
\end{itemize}
}

\subsection*{Major Comment 2: New Formulation Details}

\reviewercomment{The paper also considers a slight variation of problem formulation (2), presented in (3). The authors should provide more details on this new formulation and explain how it differs from the classical one.}

\response{We have added detailed comparison between formulations.}

\changes{Added comprehensive explanation in the Problem Formulation Clarifications remark comparing classical vs. our formulation, including the role of nonlinear inequality representations of uncertainty sets.}

\subsection*{Major Comment 3: Role of $c_i$ Terms}

\reviewercomment{In equation (10), to derive the Lagrangian function, the authors introduce $\lambda_i$ as multipliers. However, one could instead consider multipliers of the form $\gamma_i := c_i + \lambda_i$, which would reduce to the Lagrangian function of formulation (3). Therefore, I still do not see the novelty or specific role of the $c_i$ terms.}

\response{We have clarified why we use separate $c_i$ and $\lambda_i$ terms.}

\changes{Added detailed explanation in Problem Formulation Clarifications:
\begin{itemize}
\item $\lambda_i$ maintains classical dual variable interpretation (shadow prices)
\item $c_i$ provides independent regularization control
\item Separation enables Lyapunov function construction
\item Allows asymptotic recovery of classical formulation
\end{itemize}
}

\subsection*{Major Comment 4: Non-smoothness from Maximum}

\reviewercomment{In formulation (2), the authors take the maximum over the constraint functions, which significantly increases the problem's complexity compared to the classical robust optimization problem (1). Specifically, if the constraint functions are smooth, taking the maximum introduces non-smoothness, making the problem harder to solve than formulation (1).}

\response{We have added comprehensive discussion of non-smoothness handling.}

\changes{Added section on ``Non-smoothness from maximum operations'' explaining:
\begin{itemize}
\item How our dynamical approach naturally handles non-smoothness
\item Dual decomposition into smooth subproblems
\item Advantages over subgradient methods
\item Continuous-time dynamics averaging effects
\end{itemize}
}

\subsection*{Major Comment 5: Lemma 1 Novelty}

\reviewercomment{Lemma 1 is a well-known result, or am I missing something? Since Definition 1 KKT conditions seems to be identical to the KKT conditions, which are then referred to as a saddle point condition, when certain constraint qualifications hold.}

\response{We have clarified the novelty of Lemma 1.}

\changes{Enhanced the existing remark on ``Novelty of the Saddle Point Property'' explaining that while the result appears standard, its application to our non-convex-concave Lagrangian with the specific min-max-max-min structure is novel and requires careful analysis.}

\subsection*{Minor Comment 1: Formulation Generality}

\reviewercomment{Note that even if $U_0$ is a singleton, formulation (2) is defined over general sets $U_i$, whereas in formulation (3), the variables $u_i$ are specifically defined over the convex functions $h_{i,j}$.}

\response{We acknowledge this observation and have clarified that our formulation maintains generality through the convex constraint representation.}

\changes{The clarification is included in the problem formulation section.}

\subsection*{Minor Comment 2: Set Compactness}

\reviewercomment{Is the set $U_i$ in formulation (3) still compact under the assumption that the functions $h_{i,j}$ are convex? I believe some continuity assumptions are also needed.}

\response{We have added detailed discussion on set compactness.}

\changes{Added ``Remark on Set Compactness under Convex Assumptions'' explaining:
\begin{itemize}
\item How compactness is preserved with convex constraints
\item Role of continuity assumptions (already in Assumption 1)
\item Practical examples (ellipsoidal, polyhedral sets)
\end{itemize}
}

\subsection*{Minor Comment 3: Assumption 3 Restriction}

\reviewercomment{Assumption 3 states that $c_i > 0$, which implies that the problem formulation presented in (2) is not recovered. Since this assumption is introduced for technical reasons, it may need to be relaxed to ensure consistency with formulation (2).}

\response{We have added comprehensive discussion on relaxing Assumption 3.}

\changes{Added ``Remark on Justification and Relaxation of Assumption 3'' explaining:
\begin{itemize}
\item Technical justification for $c_i > 0$
\item How to choose small $c_i$ values in practice
\item Conditions under which $c_i = 0$ is allowed
\item Adaptive strategies for $c_i$ selection
\end{itemize}
}

\subsection*{Minor Comment 4: Nonlinear Constraints}

\reviewercomment{In the numerical experiments, the constraints $f_i(x, u_i)$ are linear in $u_i$. Did the authors observe similar results when dealing with problems such that the constraints are not necessarily linear in $u_i$?}

\response{We have added examples with nonlinear constraints.}

\changes{
\begin{itemize}
\item Added new example: ``Robust Portfolio Optimization with Nonlinear Constraints''
\item Added ``Remark on Extension to Nonlinear Constraints'' explaining theoretical guarantees
\item Demonstrated effectiveness with nonlinear constraint functions
\end{itemize}
}

\section*{Response to Reviewer 10}

\subsection*{Comment 1: Technical Note vs. Article}

\reviewercomment{[Suggested resubmission as technical note rather than full article]}

\response{We have substantially enhanced the technical depth to justify the full article format.}

\changes{
\begin{itemize}
\item Added comprehensive convergence and stability analysis
\item Included extensive theoretical justifications
\item Added new practical examples
\item Expanded comparison with state-of-the-art methods
\item Enhanced all technical sections with deeper analysis
\end{itemize}
}

\subsection*{Comment 2: Abstract Length}

\reviewercomment{[Abstract should be shortened]}

\response{We have significantly shortened the abstract.}

\changes{Reduced abstract from 15+ lines to 8 concise lines while maintaining key information.}

\subsection*{Comment 3: Writing Style}

\reviewercomment{[Break up long sentences, replace ``the paper'' with ``this paper'', improve clarity]}

\response{We have improved the writing style throughout.}

\changes{
\begin{itemize}
\item Broke up long sentences for better readability
\item Replaced ``the paper'' with ``this paper'' consistently
\item Enhanced overall clarity and conciseness
\end{itemize}
}

\subsection*{Comment 4: Continuity Assumption}

\reviewercomment{[Add continuity assumption to Footnote 2]}

\response{We have added the continuity assumption.}

\changes{Added continuity requirement in the relevant footnote and Assumption 1.}

\subsection*{Comment 5: Notation Issues}

\reviewercomment{[Various notation inconsistencies and missing parentheses]}

\response{We have fixed all notation issues identified.}

\changes{
\begin{itemize}
\item Reviewed and corrected notation consistency
\item Added missing parentheses where needed
\item Ensured mathematical expressions are properly formatted
\end{itemize}
}

\subsection*{Comment 6: Lemma 1 Novelty}

\reviewercomment{[Clarify novelty of Lemma 1 - seems like standard saddle point property]}

\response{We have clarified the novelty aspects.}

\changes{Enhanced explanation of why Lemma 1, while appearing standard, requires careful analysis for our specific non-convex-concave Lagrangian structure.}

\subsection*{Comment 7: Remark 4 Clarity}

\reviewercomment{[Make Remark 4 formulation clearer]}

\response{We have completely restructured Remark 4.}

\changes{Rewrote the remark with clear explanation of why standard primal-dual dynamics fail and how our approach addresses this issue.}

\subsection*{Comment 8: Theorem 4 Proof}

\reviewercomment{[Make conclusion in Theorem 4 proof more self-evident]}

\response{We have enhanced the proof clarity.}

\changes{Added step-by-step explanation of key conclusion steps with clear logical flow from set relationships to global convergence.}

\subsection*{Comment 9: Assumption 3 Rigidity}

\reviewercomment{[Address why Assumption 3 (C > 0) is too rigid for practical modeling]}

\response{We have addressed the practical aspects of Assumption 3.}

\changes{
\begin{itemize}
\item Added empirical strategies for selecting $C$ parameters
\item Explained how small values can be used without affecting solutions
\item Discussed adaptive approaches for $C$ selection
\item Added proximal regularization extension for cases with inactive constraints
\end{itemize}
}

\subsection*{Comment 10: Missing Examples}

\reviewercomment{[Add actual robust optimization examples with scenario-based uncertain constraints]}

\response{We have added comprehensive examples.}

\changes{
\begin{itemize}
\item Added robust portfolio optimization example with nonlinear constraints
\item Included scenario-based uncertainty handling
\item Demonstrated practical applications
\end{itemize}
}

\subsection*{Comment 11: Convergence Analysis}

\reviewercomment{[Include convergence analysis results]}

\response{We have added extensive convergence analysis.}

\changes{
\begin{itemize}
\item Added Section V-D.1 on Convergence Rate Analysis
\item Added Section V-D.2 on Stability Analysis
\item Enhanced theoretical convergence properties discussion
\end{itemize}
}

\section*{Summary}

We have thoroughly addressed all reviewer comments through:
\begin{itemize}
\item Complete restructuring of the Introduction with explicit contributions
\item Addition of 15+ clarifying remarks throughout the manuscript
\item Comprehensive language and formatting corrections
\item Enhanced technical depth with convergence and stability analysis
\item New practical examples demonstrating broader applicability
\item Detailed comparisons with state-of-the-art methods
\end{itemize}

All changes are marked in {\color{blue}blue} in the revised manuscript. We believe these revisions have significantly improved the clarity, rigor, and contribution of our work. We hope the reviewers will find the revised manuscript suitable for publication.

\vspace{1em}
\noindent Sincerely,\\
The Authors

\end{document}